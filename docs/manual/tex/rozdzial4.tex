	\newpage
\section{Implementacja}		%4
%Opisać implementacje algorytmu/programu. Pokazać ciekawe fragmenty kodu
%Opisać powstałe wyniki (algorytmu/nrzędzia)

\subsection{Implementacja Algorytmu obliczania $\pi$}

Algorytm zaimplementowany jest w funkcji \texttt{getPi()}, pokazanej na listingu nr.~\ref{lst:getpi} zawartej w \texttt{main.cpp}.

\begin{lstlisting}[caption=Funkcja \texttt{getPi()}, label={lst:getpi}, language=C++]
double getPi(const size_t steps, const size_t threads) {                               
	long double deltax { 1. / steps };                                               
                                                                                     
	if(deltax > 1) {                                                                 
		std::println("Rectangle width should be smaller than 1");                    
		return -1;                                                                   
	}                                                                                
                                                                                     
	if(threads == 0) {                                                               
		std::println("Threads should be a non-zero value");                          
		return -1;                                                                   
	}                                                                                
                                                                                     
	double sum = 0.;                                                                 
		                                                                             
	auto calculate = [&](const long double start, const long double end) {           
		long double localSum = 0.;                                                   
		for(long double i {start}; i < end; i+= deltax) {                            
			localSum += sqrt(1 - i * i) * deltax;                                    
		}                                                                            
                                                                                     
		sum += localSum;                                                             
	};                                                                               
                                                                                     
	std::vector<std::thread> threadPool;                                             
	threadPool.reserve(threads);                                                     
                                                                                     
	for(int i = 0; i < threads; i++) {                                               
		threadPool.push_back(std::thread([&, i] {                                    
			calculate(                                                               
				(1. / static_cast<double>(threads)) * i,                             
				(1. / static_cast<double>(threads)) * (i + 1)                        
			);                                                                       
		}));                                                                         
                                                                                     
	}                                                                                
                                                                                     
	for(auto &thread : threadPool) {                                                 
		thread.join();                                                               
	}                                                                                
                                                                                     
	return sum * 4;                                                                  
}                                                                                    
\end{lstlisting}

Na linijce nr.~1 zawarta jest deklaracja funkcji \texttt{getPi()}. Parametr \texttt{steps} określa gęstość prostokątów w ćwiartce, czyli dokładność obliczeń. Parametr \texttt{threads} natomiast, określa liczbę wątków jaka ma być używana podczas działania algorytmu. W linijce nr.~2, parametr \texttt{deltax} to szerokość poszczególnego prostokąta. W linijce nr.~4 i nr.~9 sprawdzana jest poprawność parametrów. W linijce nr.~14 zadeklarowana jest zmienna \texttt{sum}, która ma przechowywać sumę pól prostokątów. W linijce nr.~16 zadeklarowana jest funkcja lambda \texttt{calculate()}, która oblicza sumę pól prostokątów dla danego przedziału. Przedział określony jest przez parametry \texttt{start} i \texttt{end}. Na końcu lambdy, po zsumowaniu pól zmienna \texttt{localSum} dodawana jest do zmiennej \texttt{sum} zbieranej przez referencję. Od linijek nr.~25-34 tworzony jest wektor \texttt{threadPool} i wypełniany jest on wątkami, które po stworzeniu wywołują lambdę \texttt{calculate()}. Na końcu, w linijkach nr.~38-40, czekamy na zakończenie wszystkich wątków przy użyciu metody \texttt{.join()}. Blokuje ona główny wątek, aż wykonywanie odpowiedniego wątku się nie zakończy. Na końcu, zwracamy $4 \times \texttt{sum}$, czyli pole, a dlatego że przyjęte jest, że $r = 1$, jest to $\pi$.

\subsection{Implementacja kodu benchmarkującego}

Funkcja ta też znajduje się w pliku \texttt{main.cpp}. Celem jest generowanie pliku \texttt{.csv} z wynikami benchmarków.

\begin{lstlisting}[caption=Funkcja \texttt{generateBenchmarks()}, label={lst:generatebenchmarks}, language=C++,literate={µ}{$\upmu$}1]
void generateBenchmarks(
	const size_t minThreads,
	const size_t maxThreads,
	const size_t minIter,
	const size_t maxIter
) {
	std::ofstream output;
	output.open("benchmark.csv", std::ios::trunc | std::ios::out);
	
	if(!output.is_open()) {
		std::cerr << "Couldn't open benchmark file";
		return;
	}

	output << "Threads;Iterations;Calculated;Time[µs]" << std::endl;

	for(auto threads {minThreads}; threads <= maxThreads; threads++) {
		size_t multiplier {10};

		for(
			size_t iterations {minIter};
			iterations <= maxIter;
			iterations >= 1'000'000'000 ? iterations += 1'000'000'000 : iterations *= multiplier
		) {
			std::println("Benchmarking with {} threads and {} iterations.", threads, iterations);

			auto benchmarkStart { std::chrono::high_resolution_clock::now() };
			auto calculated = getPi(iterations, threads);
			auto benchmarkStop { std::chrono::high_resolution_clock::now() };

			auto benchmarkDuration = std::chrono::duration_cast<std::chrono::microseconds>(
				benchmarkStop - benchmarkStart
			);

			output << std::format("{};{};{};{}", threads, iterations, calculated, benchmarkDuration.count()) << std::endl;
		}
	}
}
\end{lstlisting}
% minThreads,
% maxThreads,
% minIter,   
% maxIter    
Jak widać na listingu nr.~\ref{lst:generatebenchmarks}, na linijkach nr.~1-6 funkcja przyjmuje parametry \texttt{minThreads}, określający minimalną liczbę wątków, \texttt{maxThreads}, określający maksymalną liczbę wątków, \texttt{minIter}, określający minimalną liczbę iteracji, \texttt{maxIter}, określający maksymalną liczbę iteracji. Na linijkach nr.~8-14 otwieramy plik \texttt{benchmark.csv} do zapisu. Jest to plik \texttt{.csv} ze strukturą \texttt{Threads;Iterations;Calculated;Time[µs]}. \texttt{Theads} określa liczbę wątków używaną podczas testu, \texttt{Iterations} liczbę iteracji, \texttt{Calculated} to wartość wykalkulowanej liczby $\pi$, a \texttt{Time[µs]} to czas pojedynczego pomiaru. 

Na linijce nr.~17, zewnętrzna pętla \texttt{for} powtarza się \texttt{maxThreads - minThreads} razy. To dlatego, że chcemy uzyskać pomiary dla wszystkich ilości wątków w danym przedziale. Zadeklarowana na linijce nr.~18 zmienna \texttt{multiplier} określa jak szybko należy zmieniać liczbę iteracji w funkcji \texttt{getPi()}. Mniejsze wartości \texttt{multiplier} pozwalają na większą dokładność, kosztem czasu wykonywania benchmarków. Na linijce nr.~20, wewnętrzna pętla \texttt{for} manipuluje ilością iteracji jaka ma być wykonywana. Maksymalnie może być to \texttt{maxIter} razy i minimalnie \texttt{minIter}, ale licznik \texttt{iterations} przy każdej iteracji jest mnożony razy 10 oraz jeżeli jest większy niż \texttt{1'000'000'000}, to dodawane jest \texttt{1'000'000'000}, ku celu oszczędzenia czasu. Następnie, przy użyciu biblioteki \texttt{std::chrono} mierzony jest czas przed i po wykonaniu funkcji oraz zapisywany jest czas trwania. Na linijce nr.~35 zapisywane są wyniki do pliku. 
